\relax 
\@writefile{toc}{\contentsline {chapter}{\numberline {A}Formats}{77}}
\@writefile{lof}{\addvspace {10\p@ }}
\@writefile{lot}{\addvspace {10\p@ }}
\@writefile{toc}{\contentsline {section}{\numberline {A.1}Grilles de calibration}{77}}
\newlabel{Format:Grille}{{A.1}{77}}
\@writefile{toc}{\contentsline {subsection}{\numberline {A.1.1}Format de codage des d\'eformations du plan }{77}}
\@writefile{toc}{\contentsline {subsubsection}{\numberline {A.1.1.1}Format}{77}}
\newlabel{GRID:TABUL:X}{{A.1}{77}}
\newlabel{GRID:TABUL:Y}{{A.2}{77}}
\@writefile{toc}{\contentsline {subsubsection}{\numberline {A.1.1.2}Utilisation}{78}}
\newlabel{Util:Grid}{{A.1.1.2}{78}}
\@writefile{toc}{\contentsline {subsection}{\numberline {A.1.2}Application \`a la calibration interne}{78}}
\@writefile{toc}{\contentsline {subsubsection}{\numberline {A.1.2.1}Rappels et notation}{78}}
\newlabel{Proj:Cam:Ideale}{{A.4}{78}}
\newlabel{MODEL:RADIAL}{{A.7}{78}}
\newlabel{MODEL:DECENTRE}{{A.9}{79}}
\newlabel{Modele:Fraser}{{A.10}{79}}
\newlabel{EQ:DIST:GEN}{{A.12}{79}}
\@writefile{toc}{\contentsline {subsubsection}{\numberline {A.1.2.2}Codage des distorsion par des grilles}{79}}
\@writefile{toc}{\contentsline {subsubsection}{\numberline {A.1.2.3}Justification des mod\`eles parm\'etriques}{80}}
\@writefile{toc}{\contentsline {subsection}{\numberline {A.1.3}Application \`a une rotation pr\`es}{81}}
\@writefile{toc}{\contentsline {subsubsection}{\numberline {A.1.3.1}Formalisation}{81}}
\newlabel{DIST:ROT:PRES}{{A.1.3.1}{81}}
\newlabel{EQ:PI0RP}{{A.33}{82}}
\newlabel{EQ:PROJ:NON:AMB}{{A.36}{82}}
\newlabel{EQ:PROJ:AMB}{{A.38}{82}}
\@writefile{toc}{\contentsline {subsubsection}{\numberline {A.1.3.2}Cons\'equence pour la comparaison de calibration}{82}}
\@writefile{toc}{\contentsline {subsection}{\numberline {A.1.4}Point principal}{83}}
\@writefile{toc}{\contentsline {subsubsection}{\numberline {A.1.4.1}Le point principal bouge}{83}}
\newlabel{DIST:HORS:GR}{{A.1.4.1}{83}}
\@writefile{toc}{\contentsline {subsubsection}{\numberline {A.1.4.2}On a perdu le point principal}{83}}
\@writefile{toc}{\contentsline {subsubsection}{\numberline {A.1.4.3}On a retrouv\'e le point principal ?}{84}}
\@writefile{toc}{\contentsline {subsection}{\numberline {A.1.5}Param\`etres hors grilles}{84}}
\@writefile{toc}{\contentsline {subsubsection}{\numberline {A.1.5.1}Param\`etres photogram\'etrique "traditionnel"}{84}}
\@writefile{toc}{\contentsline {subsubsection}{\numberline {A.1.5.2}Autres param\`etres}{84}}
\@writefile{toc}{\contentsline {section}{\numberline {A.2}Points Homologues}{84}}
\newlabel{Format:Homol}{{A.2}{84}}
\@writefile{toc}{\contentsline {section}{\numberline {A.3}Fichiers MNT}{84}}
\newlabel{Format:MNT}{{A.3}{84}}
\@setckpt{Annexes/Formats}{
\setcounter{page}{85}
\setcounter{equation}{41}
\setcounter{enumi}{3}
\setcounter{enumii}{0}
\setcounter{enumiii}{0}
\setcounter{enumiv}{0}
\setcounter{footnote}{3}
\setcounter{mpfootnote}{0}
\setcounter{part}{4}
\setcounter{chapter}{1}
\setcounter{section}{3}
\setcounter{subsection}{0}
\setcounter{subsubsection}{2}
\setcounter{paragraph}{0}
\setcounter{subparagraph}{0}
\setcounter{figure}{0}
\setcounter{table}{0}
\setcounter{parentequation}{0}
\setcounter{r@tfl@t}{0}
}
