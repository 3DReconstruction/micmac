\chapter{XML-Formal Parameter Specification}


\label{Mic:Tree:Match}

%-------------------------------------------------------------------
%-------------------------------------------------------------------
%-------------------------------------------------------------------

\section{Introduction}

\subsection{General Mechanism}

This chapter describes the \emph{formal} parameter specification of  "complex"
tools as qualified in~\ref{Gen:ExTools} (i.e. {\tt Apero,
MicMac, Porto and Casa} for now).  The general idea is that for each of these tools:

\begin{itemize}
   \item the user defines the parameters of the tool essentially by giving a file containing an XML-tree;

   \item there is, in a given file, an XML structure that describes the XML
         trees \UNCLEAR{that are valid parameters};

   \item these specification trees are all located in the directory {\tt include/XML\_GEN/};
         
   \item for  tools like {\tt Apero}, {\tt MicMac}, there is a
         file  (i.e. {\tt ParamMICMAC.xml} and {\tt ParamApero.xml}) 
         dedicated  to the specifications; 

    \item for  other tools, the specifications  are  distributed in
          files that  contain several specifications 
          \newline ({\tt ParamChantierPhotogram.xml} and {\tt SuperposImage.xml}).
          For example, the file {\tt SuperposImage.xml} contains the tree 
          {\tt <CreateOrtho>} that specifies the ortho-photo  tool  {\tt Porto}
          and the tree {\tt <CreateCompColoree>} that specifies the registration 
          superposition tool {\tt CompColore};

   \item the {\tt MicMac-LocalChantierDescripteur.xml} (that will be described in~\ref{Chap:NFS})
         is a file  that describes the common characteristics of a "project", its specification
         is described in  {\tt ParamChantierPhotogram.xml}.
\end{itemize}


These  specifications are said formal because they specify the syntax and the 
typing but not the "semantic". For example, they can specify that in certain \UNCLEAR{contains}
there must exist unique tags {\tt <X>} and {\tt <Y>} whose values must be valid
integers, but they cannot specify that these values must respect $X>Y$.

\subsection{Format Specification}

This {\tt XML} specification mechanism is important to understand because it is not 
only \UNCLEAR{parameters specification}, but also used as a format specification
of most of the data used by different tools as input or output.

For example, some programs of this tools use cylinders. They must
be able to read and write a cylinder on a file to communicate between them.
In some occasions, the user may need to provide its own cylinder as input to one
of the program. The specification of the {\tt XML} encoding of cylinder is
in the file {\tt SuperposImage.xml} where the
structure {\tt XmlCylindreRevolution} is specified. In this file the user can
find the line \footnote{here, a cylinder is coded by three points, two
for the axis and one on the cylinder fixing the radius and the cylindrical coordinates origins}:

\begin{verbatim}
<!-- P0, P1 : axe of the cylinder, POnCyl : a point of the cylinder
     fixing radius and origins of cylindrics coordinates -->

    <XmlCylindreRevolution   Nb="1"  Class="true" ToReference="true">
        <!-- P0-P1 , deux points sur la droite -->
         <P0 Nb="1" Type="Pt3dr">  </P0>
         <P1 Nb="1" Type="Pt3dr">  </P1>
         <POnCyl  Nb="1" Type="Pt3dr"> </POnCyl>
    </XmlCylindreRevolution>
\end{verbatim}

With these specifications, and some experience, the user could easily
understand that he can create a vertical cylinder with axis
on point {\tt (10,10)} of radius 5 by:


\begin{verbatim}

    <XmlCylindreRevolution>
         <P0 >  10 10 -1  </P0>
         <P1 >  10 10 1  </P1>
         <POnCyl> 13 14 25.4  </POnCyl>
    </XmlCylindreRevolution>
\end{verbatim}


Obviously, as for the parameters, this only specifies the syntax, not the
semantic. By the way, for the most simple types, the syntax and the bit
of semantic encoded in the tags name is often sufficient for understanding
the specification.


\subsection{Command Line}

\label{UTIL:SCPECIF:LIGCOM}

For the tools using XML setting, users must specify a mandatory
file  name followed by any number (generally $0$) \UNCLEAR{additional %and additional parameters 
parameters}. For example, with MICMAC, the syntax will be:


\vspace{0.2cm}
{\tt MICMAC FILE.xml Tag$_0$=Val$_0$ 
    Tag$_1$=Val$_1$  \dots   Tag$_N$=Val$_N$ }
\vspace{0.2cm}


With:

\begin{itemize}
    \item  {\tt FILE.xml}, name of an existing file containing a valid
           {\tt <ParamMICMAC>} tree;
    \item $N$ pairs {\tt (Tag$_k$,Val$_k$)} where {\tt Tag$_k$} 
          is the name of an XML {\tt Tag} and {\tt Val$_k$} is the
           associated  value;

     \item  Pairs {\tt Tag-Values} of command line  allow  the 
            user to \UNCLEAR{tune} %adjust
              the  {\tt XML} structure parameter without
            modifying the   {\tt XML}  file;

      \item  this modification mechanism is useful for interactive tests;
             its essential utility is internal when the tools are recursively
             calling themselves (for parallelization).
     
\end{itemize}

%-------------------------------------------------------------------
%-------------------------------------------------------------------
%-------------------------------------------------------------------

\section{Tree Matching}

This section describes the tree-matching process. For presentation purpose, we
suppose that {\tt MICMAC} is the program it applies to, and
{\tt ParamMICMAC.xml} is the file. Of course, it would be exactly the same
with {\tt Apero} and {\tt ParamApero.xml}, or all other tools
using {\tt XML} file parameters.

The files {\tt ParamChantierPhotogram.xml} and {\tt SuperposImage.xml} contain 
specifications that may be used by all the tools.


\label{FR:Mic:Tree:Match}

MICMAC uses a tree-matching notion to specify which are the files with valid parameters:
the tree contained in {\tt Fichier.xml} must be complementary to the tree contained in 
the \UNCLEAR{kinship} of the {\tt <ParamMICMAC>} tag from {\tt ParamMICMAC.xml}.

Some definitions are necessary for being able to specify what MICMAC considers
a valid tree-matching:
\begin{itemize}
    \item the notions of trees, \UNCLEAR{sons} %leaves 
    and \UNCLEAR{bottom-level nodes} are considered to be known;
    \item the trees from {\tt ParamMICMAC.xml} are specification trees and
    the ones from {\tt Fichier.xml} are the \UNCLEAR{real} trees;
    \item the specification trees they all have an \UNCLEAR{attribute} called the arity 
    \UNCLEAR{attribute}.
\end{itemize}

\begin{comment}
MICMAC utilise une notion d'appariement d'arbres
pour  sp\'ecifier quels sont les fichiers de param\`etres valides :
l'arbre contenu dans {\tt Fichier.xml} doit \^etre appariable sur 
l'arbre contenu dans la descendance du  tag {\tt <ParamMICMAC>} contenu
dans la ressource {\tt ParamMICMAC.xml}.

Quelques d\'efinitions sont n\'ecessaires pour pouvoir pr\'eciser 
ce que MICMAC consid\`ere comme un appariement d'arbre valide :

\begin{itemize}
   \item les notions d'arbres, de fils et de n\oe{}uds terminaux sont suppos\'ees
         connues;
   \item  on appelle arbre de sp\'ecification les arbres de 
          {\tt ParamMICMAC.xml}  et arbre effectif les arbres de  
          {\tt Fichier.xml};
   \item les arbres de sp\'ecification ont tous un attibut dit attribut d'arit\'e;
         cet attribut, d\'esign\'e par le mot-clef $Nb$ peut valoir $1$,
         $?$, $*$ ou $+$; ce mot cl\'e repr\'esente un intervalle d'entier
         selon les conventions classiques : $1$=exactement un, $?$= un ou z\'ero,
         $+$=un au moins, $*$= un nombre quelconque;
    
\end{itemize}
\end{comment}


\begin{center}
    \emph{ TO FRENCH}
\end{center}



Un arbre effectif est appariable sur un arbre de sp\'ecification si et
seulement si :

\begin{itemize}
   \item ils ont le m\^eme nom de tag;
   \item chaque fils, de l'arbre effectif est appariable sur un  fils de l'arbre
         de sp\'ecification;
   \item chaque fils de l'arbre de sp\'ecification est appariable
         sur un ensemble de $N$ fils de l'arbre
         effectif,  o\`u $N$ est compatible avec l'arit\'e de l'arbre de sp\'ecification
         (cette d\'efinition implique que, par exemple, si $Nb=?$, l'ensemble des
           n\oe{}uds  appariables de l'arbre effectif peut \^etre vide);
   \item  lorsque le  n\oe{}ud est terminal, le  n\oe{}ud de l'arbre de sp\'ecification
          porte un attribut {\tt Type} qui impose \'eventuellement des contraintes sur
          la valeur du n\oe{}ud  de l'arbre effectif;
\end{itemize}

On notera que, pour l'instant du moins, l'ordre des fils n'importe
pas pour d\'efinir si deux arbres sont appariables.  Il est 
d\'econseill\'e d'utiliser cette propri\'et\'e (des versions 
ult\'erieure pourront imposer une conservation de l'ordre).

Soit par exemple l'arbre de sp\'ecification suivant :

\begin{verbatim}
<UnTestParam>
     <ListTestCpleHomol Nb="*">
                <PtIm1 Type="Pt2di" Nb="1" > </PtIm1>
                <PtIm2 Type="Pt2di" Nb="1" > </PtIm2>
     </ListTestCpleHomol>
     <NomModule Nb="1" Type="std::string"> </NomModule>
     <NomAux Nb="?" Type="std::string">    </NomAux>
</UnTestParam>
\end{verbatim}



Les deux arbres effectifs ci dessous sont appariables sur l'exemple
pr\'ec\'edent :

\begin{verbatim}

<UnTestParam>
     <ListTestCpleHomol>
           <PtIm1> 1 2 </PtIm1>
           <PtIm2> 2 3  </PtIm2>
     </ListTestCpleHomol>
     <NomModule > UnNom  </NomModule>
     <ListTestCpleHomol>
           <PtIm1> 0 0 </PtIm1>
           <PtIm2> 0 0  </PtIm2>
     </ListTestCpleHomol>
</UnTestParam>

<UnTestParam>
     <NomModule > UnNom  </NomModule>
     <NomAux >   UnAutreNom  </NomAux>
</UnTestParam>

\end{verbatim}

 

L'arbre effectif ci dessous n'est pas  appariable
sur l'arbre de sp\'ecification pour (au moins) les
raisons suivantes :

\begin{itemize}
   \item dans l'arbre de sp\'ecification {\tt <UnTestParam>} 
         n'a pas de fil nomm\'e {\tt  <Toto>} ;
   \item dans l'arbre effectif, le contenu de {\tt <PtIm1>} et
         {\tt <PtIm2>}  ne correspondent pas \`a ce qui est attendu pour
         le type {\tt Pt2di} (ce qui signifie \emph{Point 2D entier},
         et attend de lire deux valeurs enti\`ere);
    \item dans l'arbre de sp\'ecification {\tt <UnTestParam>}
          a un fils nomm\'e  {\tt  <NomModule>}, d'arit\'e $1$ qui
          ne se retouve pas dans l'arbre effectif;
    \item  {\tt <NomAux>} n'a pas la bonne arit\'e ($2$ pour 
           $0$ ou $1$ autoris\'e);
\end{itemize}

\begin{verbatim}

<UnTestParam>
     <Toto> n'a pas d'homologue </Toto>
     <ListTestCpleHomol>
           <PtIm1> 1 2 3 </PtIm1>
           <PtIm2> 2   </PtIm2>
     </ListTestCpleHomol>
     <NomAux >   UnAutreNom  </NomAux>
     <NomAux >   UnAutreNom  </NomAux>
</UnTestParam>

\end{verbatim}

%-------------------------------------------------------------------
%-------------------------------------------------------------------
%-------------------------------------------------------------------

\section{Types des n\oe{}uds terminaux}

\subsection{Types g\'en\'eraux}

Dans l'arbre de sp\'ecification chaque noeud terminal comporte un attribut
de type nomm\'e {\tt Type}. Cet attribut {\tt Type} est utilis\'e, lors de la g\'en\'eration
automatique de code (voir chapitre~\ref{PROG:GENEAUTO}), pour
d\'eterminer le type de la classe  \CPP qui doit \^etre utilis\'ee
pour repr\'esenter la valeur contenue dans le n\oe{}ud appari\'e de 
l'arbre effectif.


Cet attribut {\tt Type}   d\'etermine aussi si la valeur (=cha\^ine de 
caract\`eres) contenue dans un champ est valide. Les type existant et
leur pr\'e-requis sur les valeurs sont :


\begin{itemize}
   \item {\bf \tt std::string} aucune restriction sur la valeur;
   \item {\bf \tt bool} la valeur doit \^etre dans $\{0,1,true,false\}$
         ($0$ \'etant \'equivalent \`a $false$ et $1$ \`a $true$);
   \item {\bf \tt int} un seul entier;
   \item {\bf \tt double} un seul r\'eel;
   \item {\bf \tt std::vector<double>} un nombre quelconque de r\'eels;
   \item {\bf \tt std::vector<int>} un nombre quelconque d'entiers;
   \item {\bf \tt Pt2di} deux entiers ($x$ et $y$);
   \item {\bf \tt Pt2dr}  deux r\'eels ($x$ et $y$);
   \item {\bf \tt Pt3dr}  trois r\'eels ($x$, $y$ et $z$);
   \item {\bf \tt Box2dr} quatre r\'eels ($x_0 \ y_0\ x_1\ y_1$);
   \item les types \'enum\'er\'es, qui sont d\'ecrits \`a la section
         suivante, et pour lesquels la valeur doit appartenir \`a 
         l'ensemble des valeurs \'enum\'er\'ees par le type;
\end{itemize}

Lorsque le type terminal est un vecteur ({\tt int} ou {\tt double} pour
l'instant), la syntaxe est "entre crochet avec virgule comme s\'eparateur",
par exemple  avec un double {\tt [1,2.3,4]}.

\subsection{Types \'enum\'er\'es}

\label{Type::Enum}

Dans le fichier de sp\'ecification {\tt ParamMICMAC.xml},
avant la d\'efinition de l'arbre de sp\'ecification {\tt <ParamMICMAC>},
figurent la d\'efinition des types  \'enum\'er\'es.

La d\'efinition d'un type \'enum\'er\'e, de nom {\tt UnType}, et
pouvant prendre les valeur  {\tt Val-1} {\tt Val-2} \dots {\tt Val-N}
se fait selon la syntaxe :

\begin{verbatim}
  <enum Name="UnType">
         <Val-1> </Val-1>
         <Val-2> </Val-2>
               ...
         <Val-N> </Val-N>
   </enum>

\end{verbatim}


Les types \'enum\'er\'es n'ont pas forc\'ement \'et\'e d\'efini dans {\tt ParamMICMAC.xml}.
Lorsqu'ils ont une utilisation en dehors de {\tt MICMAC}, ils sont d\'efinis dans un
des fichiers  g\'en\'eraux ( {\tt ParamChantierPhotogram.xml} \dots). Par exemple
le type \'enum\'er\'e {\tt eModeGeomMNT} sp\'ecifie le tag {\tt <GeomMNT>}
dans le fichier {\tt ParamMICMAC.xml}, mais il est d\'efini dans 
dans {\tt ParamChantierPhotogram.xml} (parce qu'il est utilis\'e plusieurs
fois et dans plusieurs fichiers : 
{\tt ParamChantierPhotogram.xml, ParamMICMAC.xml, SuperposImage.xml}).



%-------------------------------------------------------------------
%-------------------------------------------------------------------
%-------------------------------------------------------------------

\section{D\'efinition de types d'arbres}

\label{Def:Type:Arbre}

Lorsque un m\^eme type d'arbre est utilis\'e plusieurs fois,
MicMac utilise en g\'en\'eral un m\'ecanisme de d\'efinition
de type qui peut \^etre ensuite r\'ef\'erenc\'e.

Lorsque de la   d\'efinition du type d'arbre , l'attribut {\tt ToReference}
est \`a {\tt true}. Ensuite, lors de l'utilisation ,
l'attribut {\tt RefType} a pour valeur le nom du
type \`a r\'ef\'erencer.

Par exemple on pourrait avoir :

\begin{verbatim}
   <Personne Nb="1" Class="true" ToReference="true">
        <Age Nb="1" Type="double">   </Age>
        <Nom Nb="1" Type="std::string">   </Nom>
   </Personne>
...
   <Club Nb="1">
       <Membre Nb="*" RefType="Personne"> </Membre>
       <NomClub Nb="1" Type="std::string">  </NomClub>
   </Club>
...
\end{verbatim}

Et une utilisation correcte :

\begin{verbatim}
  <Club>
      <NomClub> Contributeurs  MicMac </NomClub>
      <Membres> <Age> 45 </Age> <Nom> MPD </Nom> </Membres>
      <Membres> <Age> 35 </Age> <Nom> GMaillet </Nom> </Membres>
      <Membres> <Age> 32 </Age> <Nom> DBoldo </Nom> </Membres>
  </Club>
\end{verbatim}

Les types qui sont utilis\'es par plusieurs outils sont d\'efinis
dans un fichier g\'en\'eral et ensuite utilis\'es plusieurs fois
dans les diff\'erents outils. Lorsque le fichier de d\'efinition est
diff\'erent de l'utilisation, on trouvera dans le tag d'utilisation
un attribut {\tt RefFile} indiquant le fichier de d\'efinition
\footnote{ceci n'existe pas pour les type \'enum\'er\'es}.
Par exemple :


\begin{itemize}
   \item  le type {\tt ChantierDescripteur} est utilis\'es par
          tous les outils un peu complexes;

   \item il est d\'efini dans {\tt ParamChantierPhotogram.xml};
   
    \item il est utilis\'e dans {\tt ParamApero.xml}, sous le tag
          {\tt DicoLoc} et dans {\tt ParamMICMAC.xml} sous le
          tag  {\tt DicoLoc}  (mais il n'y a rien d'obligatoire \`a
          ce qu'\`a l'utilisation le nom soit toujours le m\^eme);

\end{itemize}

Dans  {\tt ParamApero.xml} on trouvera :


\begin{verbatim}
       <DicoLoc  Nb="?" RefType="ChantierDescripteur"
                         RefFile="ParamChantierPhotogram.xml"
        >
        </DicoLoc>

\end{verbatim}



%-------------------------------------------------------------------
%-------------------------------------------------------------------
%-------------------------------------------------------------------

\section{Mode "diff\'erentiel"}

Cette section d\'ecrit un fonctionnement assez sp\'ecifique,
qui n'est utilis\'e aujourd'hui que pour le param\'etrage 
de MicMac et sur un seul de ses tag. Cependant, il est
tr\`es important dans ce cas particulier.

\label{DUG:Mode:Dif}

Un n\oe{}ud de l'arbre de sp\'ecification, d'arit\'e $+$,
peut avoir un attribut {\tt DeltaPrec} \`a $1$. Ce n'est le cas aujourd'hui
que du n\oe{}ud {\tt <EtapeMEC>} correspondant aux \'etapes de la
mise en correspondance, mais celui-ci joue un 
r\^ole essentiel dans le param\'etrage algorithmique.
Lorsque {\tt DeltaPrec=1} (par d\'efaut il vaut $0$), l`\'evaluation
de la liste  \footnote{une liste de valeur car l'arit\'e $+$, 
indique un nombre quelconque $\geq 1$}
des valeurs de l'arbre effectif se fait en mode dit diff\'erentiel.

Le mode diff\'erentiel a \'et\'e ajout\'e  pour tenir compte des situtations
o\`u, en g\'en\'eral, la liste des valeurs est "grande"
avec une forte corr\'elation entre les
valeurs successives. C'est le cas par exemple de 
{\tt <EtapeMEC>} o\`u souvent , d'une \'etape \`a l'autre,
tous les param\`etres algorithmiques sont conserv\'es et 
seule la r\'esolution change. L'objectif est alors d'offrir
un syntaxe, qui sans perdre de g\'en\'eralit\'e, permette dans
les cas les plus courants de ne sp\'ecifier que les attributs qui
diff\`erent de l'\'etape pr\'ec\'edente. Formellement, le fonctionnement
est le suivant :

\begin{itemize}
   \item le premier \'element de la liste effective ne fera pas partie
         du r\'esultat utilis\'e par MICMAC, il constitue l'initialisation
         de la \emph{valeur courante};
    \item pour les \'el\'ements suivant, la valeur rajout\'ee
          est constitu\'ee de la valeur courante, modifi\'ee des 
          fils qui sont explicit\'es dans la liste effective;
     \item les fils explicit\'es dans la liste effective ne
            modifie la valeur courant que si ils ont l'attribut
           {\tt Portee} qui a la valeur {\tt ="Globale"};
\end{itemize}

Soit par exemple l'arbre de sp\'ecification suivant :

\begin{verbatim}
<ListeNMP>
    <NMP  Nb="+" DeltaPrec="1">
        <Num  Nb="1" Type="int">  </Num>
        <Mes  Nb="?" Type="std::string"> </Mes>
        <Pi   Nb="?" Type="double">     </Pi>
    </NMP>
</ListeNMP>
\end{verbatim}

Avec l'arbre effectif de la figure~\ref{DUG:ModDiff:AEff}
c'est la liste  de la figure~\ref{DUG:ModDiff:ResulMem}
qui sera utilis\'ee par MICMAC.


\begin{figure}
\begin{verbatim}
 <ListeNMP>
    <NMP> <Num>-1</Num>  <Pi >3.0</Pi> </NMP>

    <NMP> <Num>0</Num> <Mes>Bonjour</Mes> </NMP>
    <NMP> <Num>1</Num> <Pi Portee="Globale">3.14</Pi> </NMP>
    <NMP> <Num>2</Num> <Mes>Au Revoir</Mes> </NMP>
</ListeNMP>
\end{verbatim}
\caption {Exemple d'utilisation du mode diff\'erentiel }
\label{DUG:ModDiff:AEff}

\end{figure}


\begin{figure}
\begin{verbatim}
     {Num=0 ; Pi=3.0  ; Mes="Bonjour";}
     {Num=1 ; Pi=3.14  ;}
     {Num=2 ; Pi=3.14  ; Mes="Au Revoir";}
\end{verbatim}
\caption {R\'esultat de l'utilisation du mode diff\'erentiel }
\label{DUG:ModDiff:ResulMem}
\end{figure}

On remarque que :

\begin{itemize}

   \item la  premi\`ere valeur n'est pas ajout\'ee dans la liste mais
         a pour effet de donner une valeur initiale \`a {\tt <Pi>};

   \item  la modification de {\tt <Pi>} sur la valeur {\tt Num=1}
          se propage \`a la  valeur {\tt Num=2} car la modification 
          se fait avec l'attribut {\tt Portee="Globale"};

   \item  la modification de {\tt <Mes>} sur la valeur {\tt Num=0}
          ne se propage pas ({\tt <Mes>} est absent de {\tt Num=1});
\end{itemize}


%-------------------------------------------------------------------
%-------------------------------------------------------------------
%-------------------------------------------------------------------
\section{Autres caract\'eristiques}

             % - - - - - - - - - - - - - - - - - - - -

\subsection{Modification par ligne de commande}

\label{DU:Modif:LC}

Le section~\ref{UTIL:SCPECIF:LIGCOM} a indiqu\'e qu'il
\'etait possible dans certain cas de modifier la valeur
des tags par la ligne de commande.

Un arbre du fichier de sp\'ecification, de nom {\tt UnTag}  
est modifiable dans le fichier effectif par la ligne
de commande ssi :

\begin{itemize}
   \item c'est un noeud terminal;
   \item sont attribut d'arit\'e vaut $1$ ou $?$;
   \item tous ses ascendants ont un arit\'e de $1$;
\end{itemize}

Ces r\'egles, un peu restrictives, permettent d'\'eviter 
que la modification soit ambig\"ue ou qu'elle conduise \`a
d\'efinir une syntaxe trop compliqu\'ee (par exemple pour
saisir des structures imbriqu\'ee).
La valeur pass\'ee en ligne de commande \'ecrase celle
qui existait \'eventuellement dans le fichier effectif.

             % - - - - - - - - - - - - - - - - - - - -

\subsection{Valeurs par d\'efaut}

\label{Val:Def}

Lorsqu'un  n\oe{}ud de l'arbre de sp\'ecification 
est terminal, et que son symbole d'arit\'e est $?$,
alors il peut contenir un attribut de nom {\tt Def}
indiquant la valeur par d\'efaut qui sera donn\'ee
\`a ce tag s'il n'est pas sp\'ecifi\'e par l'utilisateur.


\subsection{Fichiers de param\`etres g\'en\'er\'es}


Dans le r\'epertoire  de mise en correspondance, MICMAC
g\'en\`ere deux fichiers xml :

\begin{itemize}
   \item un fichier {\tt Fichier\_Ori.xml}  qui est une simple
         recopie, caract\`ere par caract\`ere, de {\tt Fichier.xml};
         cette copie est faites pour garder une trace des param\`etres
         avec lesquels a \'et\'e lanc\'e MICMAC;

   \item un fichier {\tt Fichier\_Compl.xml}, qui est un {\tt dump}
         de la structure m\'emoire qui a \'et\'e associ\'ee 
         \`a {\tt Fichier.xml};  {\tt Fichier\_Compl.xml} diff\`ere
         de  {\tt Fichier.xml}, notamment \`a cause des valeurs
         par d\'efaut et du mode diff\'erentiel;
\end{itemize}

Ansi avec l'exemple de la figure~\ref{DUG:ModDiff:AEff},
le fichier {\tt Fichier\_Compl.xml} contiendra le texte
de la figure~\ref{DUG:ModDiff:Compl}:

\begin{figure}
\begin{verbatim}
 <ListeNMP>
    <NMP> <Num>-1</Num> <Pi >3.0</Pi> </NMP>
    <NMP> <Num>0</Num>  <Pi>3.0</Pi>  <Mes>Bonjour</Mes> </NMP>
    <NMP> <Num>1</Num>  <Pi>3.14</Pi> </NMP>
    <NMP> <Num>2</Num>  <Pi>3.14</Pi> <Mes>Au Revoir</Mes> </NMP>
</ListeNMP>
\end{verbatim}
\caption {Exemple d'utilisation du mode diff\'erentiel }
\label{DUG:ModDiff:Compl}

\end{figure}


%-------------------------------------------------------------------
%-------------------------------------------------------------------
%-------------------------------------------------------------------




